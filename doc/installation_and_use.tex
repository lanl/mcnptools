\documentclass[11pt]{article}
\usepackage[scaled=0.83]{beramono}
\usepackage{booktabs}
\usepackage{embedfile}
\usepackage[T1]{fontenc}
\usepackage[margin=1in]{geometry}
\usepackage{graphicx}
\usepackage[unicode=true,pdfusetitle]{hyperref}
\usepackage{listings}
\usepackage{longtable}
\usepackage{microtype}
\usepackage{parskip}
\usepackage{xcolor}

%%%%%%%%%%%%%%%%%%%%%%%%%%%%%%%%%%%%%%%%%%%%%%%%%%%%%%%%%%%%%%%%%%%%%%%%%%%%%%%%
% General package and document-behavior customization.

\hypersetup{ % Order matters...
  allcolors=blue,
  unicode=true,
  bookmarks=true,
  bookmarksnumbered=true,
  bookmarksopen=false,
  breaklinks=true,
  pdfborder={0 0 0},
  pdfborderstyle={},
  colorlinks=true
}

\setlength{\parindent}{0pt}
\setlength{\parskip}{11pt plus 3pt minus 3pt}

%%%%%%%%%%%%%%%%%%%%%%%%%%%%%%%%%%%%%%%%%%%%%%%%%%%%%%%%%%%%%%%%%%%%%%%%%%%%%%%%
% Code-listing customization.

\definecolor{codebackground}{HTML}{ffffff}
\definecolor{codeframe}     {HTML}{000000}
\definecolor{codecomment}   {HTML}{1b9e77} % http://colorbrewer2.org/#type=qualitative&scheme=Dark2&n=3
\definecolor{codestring}    {HTML}{7570b3} % http://colorbrewer2.org/#type=qualitative&scheme=Dark2&n=3
\definecolor{codekeyword}   {HTML}{d95f02} % http://colorbrewer2.org/#type=qualitative&scheme=Dark2&n=3
\definecolor{bgh}{HTML}{f2f2f2}

\lstset {%
  aboveskip=+0.8em,                                  % amount of space added above the listing
  belowskip=-0.5em,                                  % amount of space added below the listing
  backgroundcolor=\color{codebackground},            % choose the background color; you must add \usepackage{color} or \usepackage{xcolor}
  basicstyle=\normalsize\ttfamily,                   % the size of the fonts that are used for the code
  columns=flexible,                                  % improve column layout for non-fixed-width fonts
  commentstyle=\color{codecomment},                  % comment style
  numberstyle=\color{gray}\tiny\zebra{white}{bgh}{}, % the style that is used for the line-numbers
  rulecolor=\color{codeframe},                       % if not set, the frame-color may be changed on line-breaks within not-black text (e.g. comments (green here))
  stringstyle=\color{codestring},                    % string literal style
  breakatwhitespace=false,                           % sets if automatic breaks should only happen at whitespace
  breaklines=true,                                   % sets automatic line breaking
  captionpos=t,                                      % sets the caption-position to bottom
  deletekeywords={...},                              % if you want to delete keywords from the given language
  escapeinside={<@}{@>},                             % if you want to add LaTeX within your code
  extendedchars=true,                                % lets you use non-ASCII characters; for 8-bits encodings only, does not work with UTF-8
  frame=single,                                      % adds a frame around the code
  keepspaces=true,                                   % keeps spaces in text, useful for keeping indentation of code (possibly needs columns=flexible)
  keywordstyle=\color{codekeyword},                  % keyword style
  numbers=left,                                      % where to put the line-numbers; possible values are (none, left, right)
  numbersep=5pt,                                     % how far the line-numbers are from the code
  showspaces=false,                                  % show spaces everywhere adding particular underscores; it overrides 'showstringspaces'
  showstringspaces=false,                            % underline spaces within strings only
  showtabs=false,                                    % show tabs within strings adding particular underscores
  stepnumber=1,                                      % the step between two line-numbers. If it's 1, each line will be numbered
  tabsize=2,                                         % sets default tabsize to 2 spaces
}

% Configure zebra striping.
\newcommand\realnumberstyle[1]{}
\makeatletter
\newcommand{\zebra}[2]{%
    \begingroup
    \lst@basicstyle
    \ifodd\value{lstnumber}%
        \color{#1}%
    \else
        \color{#2}%
    \fi
    \ifnum\value{lstnumber}>9
        \rlap{\hspace*{\lst@numbersep}\hspace*{0.95em}%
        \color@block{\linewidth}{\ht\strutbox}{\dp\strutbox}%
        }%
    \else
        \rlap{\hspace*{\lst@numbersep}\hspace*{0.5em}%
        \color@block{\linewidth}{\ht\strutbox}{\dp\strutbox}%
        }%
    \fi
    \endgroup
}

%%%%%%%%%%%%%%%%%%%%%%%%%%%%%%%%%%%%%%%%%%%%%%%%%%%%%%%%%%%%%%%%%%%%%%%%%%%%%%%%
% Frontmatter

\title{The MCNPTools Package: Installation and Use}
\author{Cameron R. Bates, Simon R. Bolding, Colin J. Josey, Joel A. Kulesza, \\Clell J. (CJ) Solomon Jr., Anthony J. Zukaitis}
\date{August 2022}

\makeatletter
\def\blfootnote{\gdef\@thefnmark{}\@footnotetext}
\makeatother
\newcommand{\makereg}{\raisebox{1ex}{\scalebox{0.5}{®}}}
\newcommand{\trademarktext}{MCNP\makereg\ and Monte Carlo N-Particle\makereg\
are registered trademarks owned by Triad National Security, LLC, manager and
operator of Los Alamos National Laboratory. Any third party use of such
registered marks should be properly attributed to Triad National Security, LLC,
including the use of the \makereg\ designation as appropriate. Any questions
regarding licensing, proper use, and/or proper attribution of Triad National
Security, LLC  marks should be directed to trademarks@lanl.gov.}

\begin{document}

\maketitle

\section{Introduction}\label{sec:introduction}

MCNPTools\blfootnote{\trademarktext} is a C++ software library bound to Python (2
\& 3) via the Simplified Wrapper and Interface Generator (SWIG version 3.0.7).
The minimum requirements to build MCNPTools as a C++ library are the following:

\begin{itemize}
  \item a C++ compiler supporting C++11 features
  \item the CMake tool set version 3.21 or above
  \item HDF5 version 1.10.2 or above
\end{itemize}

Currently, the following compiler options are tested and supported:

\begin{itemize}
  \item GCC 8.3.0 and above on Linux and macOS
  \item MSVC 19.0 on Windows
  \item Apple Clang 7.3.0 and above on macOS
  \item Intel C++ Classic Compiler 18.0.5 and above
\end{itemize}

For the Python bindings, the following must be installed:

\begin{itemize}
  \item Python 2.7 or newer
  \item Setuptools
  \item Pip
\end{itemize}

Builds of the Python bindings have been extensively tested with the Anaconda Python distribution
(\hyperlink{https://www.anaconda.com/products/individual}{https://www.anaconda.com/products/individual}),
but have been cursorily tested with other distributions as well.

\subsection{Installing MCNPTools from a Wheel}\label{sec::installing-mcnptools-from-a-wheel}

If you would like to install MCNPTools without building it yourself, you can do so by downloading a wheel for your operating system.
Then, run:
\begin{lstlisting}
  python -m pip install mcnptools-X.Y.Z-??????.whl
\end{lstlisting}
The \texttt{??????} is a placeholder for information about the system for which the specified wheel file is built, and can include your OS and Python version.
One can add the \texttt{--user} command to install in your user Python modules if you do not wish to install system-wide, or \texttt{--prefix [path]} to select an installation directory.

Note that MCNPTools will need to be re-installed whenever you upgrade your Python major version, e.g., from 3.9.X to 3.10.X.

\subsection{Building MCNPTools}\label{sec:building-mcnptools}

Once your build environment is set up (see Section~\ref{sec:setting-up-hdf5} for tips for getting HDF5 working), create a directory to build MCNPTools.
Within the directory, run the following commands:

\begin{lstlisting}
  cmake -D CMAKE_INSTALL_PREFIX=[path to install] -D mcnptools.python_install=User [path to MCNPTools source directory]
  cmake --build . --config Release
  ctest --build-config Release
  cmake --install . --config Release
\end{lstlisting}

This will configure, build, test, and install the MCNPTools library, utilities, and Python bindings.
Testing is optional but recommended.
One should confirm all tests pass prior to installation.

The two CMake variables \texttt{CMAKE\_INSTALL\_PREFIX} and \texttt{mcnptools.python\_install} control where components of MCNPTools are installed.
The location for the library and the utilities is controlled by the variable \texttt{CMAKE\_INSTALL\_PREFIX}.
The Python bindings will be placed at \texttt{CMAKE\_INSTALL\_PREFIX/lib} and the utilities will be placed at \texttt{CMAKE\_INSTALL\_PREFIX/bin}.

The Python binding install location is controlled by \texttt{mcnptools.python\_install}, which has three options:
\begin{description}
  \item[\texttt{Global}] This will install in the current global Python module directory, and is most useful for system-wide installs or for Python virtual environments. (Default)
  \item[\texttt{User}] This will will install in the current user's Python module directory.
  This is most useful for installing without administration privileges.
  \item[\texttt{Prefix}] This will install within \texttt{CMAKE\_INSTALL\_PREFIX/lib}, which is most useful for packaging and maintaining multiple versions.
  The precise location is OS-dependent, but on Linux, the location will likely be \texttt{CMAKE\_INSTALL\_PREFIX/lib/pythonX.X/site-packages}, where \texttt{X.X} corresponds to the specific Python version used to build MCNPTools.
  In this case, you will have to add the \texttt{site-packages} path to the \texttt{PYTHONPATH} environment variable for Python to find the bindings.
\end{description}

Note that MCNPTools will need to be rebuilt and re-installed whenever you upgrade your Python version, e.g., from 3.9.X to 3.10.X.

\subsubsection{Setting Up HDF5}\label{sec:setting-up-hdf5}

Sometimes it is difficult for CMake to find a working HDF5 installation, and if it does, it may not load all the necessary libraries.

CMake can find HDF5 in 3 different ways, in order of most reliable to least reliable:
\begin{enumerate}
  \item By setting the \texttt{HDF5\_DIR} environment variable to HDF5's own CMake folder, located at \texttt{<path to HDF5 install>/share/cmake/hdf5}. This folder may not be present if HDF5 was not built using CMake.
  \item Through finding the program \texttt{h5cc} in the current environment's path.
  \item By setting the \texttt{HDF5\_ROOT} environment variable to \texttt{<path to HDF5 install>}.
\end{enumerate}

\section{MCNPTools Utilities}\label{mcnptools-utilities}

MCNPTools releases include binary utilities that facilitate common tasks or
querying MCNP output files. This section provides information regarding the use
of these utilities. The usage information presented can be obtained from all
utilities by running the utility with the \texttt{-h} or \texttt{-\/-help}
options specified.

\subsection{\texttt{lnk3dnt} Utilities}

\subsubsection{\texttt{l3d2vtk}}\label{sec:l3d2vtk}

The \texttt{l3d2vtk} utility converts LNK3DNT files to XML-based
StructuredGrid VTK (.vts) files. This can be particularly useful to MCNP
users because a LNK3DNT file can be produced as MCNP output that
represents a discretized representation of the MCNP CSG, which can then
visualized interactively in a 3-D application.

By default, \texttt{l3d2vtk} produces no standard output and writes a
\texttt{lnk3dnt.vts} file. If the \texttt{-\/-verbose} option is given,
then status is output periodically as the conversion proceeds.

This utility functions for $\left(x\right)$ (Cartesian), $\left(r\right)$
(cylindrical), $\left(r\right)$ (spherical), $\left(x,y\right)$,
$\left(r,z\right)$, $\left(r,\theta\right)$, $\left(x,y,z\right)$,
$\left(r,z,\theta\right)$ geometries. For large LNK3DNT files, this utility can
become sensitive to the computer's stack size. However, large ($\sim100$ million
zone) 3-D Cartesian files have been successfully converted and visualized
interactively.

The execution options given via the help message is given in
Listing~\ref{lst:l3d2vtk-help-message-output}.

\subsubsection{\texttt{l3dcoarsen}}\label{sec:l3dcoarsen}

The \texttt{l3dcoarsen} utility coarsens a LNK3DNT file and produces a new
LNK3DNT file. By default, the resulting LNK3DNT file with have preserved
material boundaries and the same number of mixed-material zones as the original;
however, the user may keep more or less mixed-materials in a zone if desired.

The execution options given via the help message is given in
Listing~\ref{lst:l3dcoarsen-help-message-output}.

\subsubsection{\texttt{l3dinfo}}\label{sec:l3dinfo}

The \texttt{l3dinfo} utility reports information about LNK3DNT files. By
default, \texttt{l3dinfo} reports only basic information about the LNK3DNT file:
geometry, extents, etc. If the \texttt{-\/-full} option is given, then the
material information will be read and reported in addition to the basic
information.

The execution options given via the help message is given in
Listing~\ref{lst:l3dinfo-help-message-output}.

\subsubsection{\texttt{l3dscale}}\label{sec:l3dscale}

The \texttt{l3dscale} utility linearly scales the dimensions of a
LNK3DNT file by a user-specified factor and produces a new LNK3DNT file.

The execution options given via the help message is given in
Listing~\ref{lst:l3dscale-help-message-output}.

\subsection{\texttt{mctal} Utilities}

\subsubsection{\texttt{mctal2rad}}\label{sec:mctal2rad}

The \texttt{mctal2rad} utility converts MCNP image tally results (e.g.,
\texttt{FIR}, \texttt{FIP}, etc.) in a MCTAL file into TIFF images.
Accordingly, \texttt{mctal2rad} depends on libtiff being installed and available
during compilation. The output images can be created from only the direct
detector contributions and the results can be transposed and/or scaled
logarithmically.

The execution options given via the help message is given in
Listing~\ref{lst:mctal2rad-help-message-output}.

\subsubsection{\texttt{mergemctals}}\label{sec:mergemctals}

The \texttt{mergemctals} utility statistically merges the results in
multiple MCNP MCTAL files and produces a single resulting MCTAL file.

\texttt{mergemctals} can also be compiled using Boost MPI so that MCTAL
files can be merged in parallel. All machines (e.g., back-end nodes of a
cluster) performing parallel operations must have access to the files to
be merged.

The execution options given via the help message is given in
Listing~\ref{lst:mergemctals-help-message-output}.

\subsection{\texttt{meshtal} Utilities}

\subsubsection{\texttt{mergemeshtals}}\label{sec:mergemeshtals}

The \texttt{mergemeshtals} utility statistically merges the results in multiple
MCNP Type-B MESHTAL files (i.e., those created with an \texttt{fmesh} card) and
produces a single resulting MESHTAL file. \texttt{mergemeshtals} \emph{only}
operates on column-formatted MESHTAL files.

\texttt{mergemeshtals} can also be compiled using Boost MPI so that the MESHTAL
files can be merged in parallel, though all machines (e.g., back-end nodes of a
cluster) performing parallel operations must have access to the files to be
merged.

The execution options given via the help message is given in
Listing~\ref{lst:mergemeshtals-help-message-output}.

\subsubsection{\texttt{meshtal2vtk}}\label{sec:meshtal2vtk}

The \texttt{meshtal2vtk} utility converts MCNP XYZ (Cartesian) and/or RZT
(cylindrical) MCNP mesh tally results in a MESHTAL file into XML-formatted
StructuredGrid VTK (.vts) files. These files can then be viewed in scientific
visualization applications such as ParaView or VisIt.

Data series are logically named according to any binning that exists, or if no
binning, as Tally\_Value and Tally\_Error. The user has the option of selecting
only certain tallies with the TALLY parameter shown below. If left unspecified,
all tallies are processed. Each tally is given its own .vts file.

The execution options given via the help message is given in
Listing~\ref{lst:meshtal2vtk-help-message-output}.

\section{Description of the MCNPTools
Library}\label{description-of-the-mcnptools-library}

The true power of MCNPTools is in the ability for users to write their own
custom tools and process MCNP outputs without the need to parse MCNP output
formats. Currently, three MCNP output files can be read by MCNPTools and
accessed in an object-oriented manner:

\begin{description}
  \item[MCTAL files] accessed via the \texttt{Mctal} class which in turn
    provides access to the \texttt{MctalTally} and \texttt{MctalKcode} classes.
  \item[MESHTAL files] accessed via the \texttt{Meshtal} class which in turn
    provides access to the \texttt{MeshtalTally} class
  \item[PTRAC files] accessed via the \texttt{Ptrac} class which in turn
    provides access to the \texttt{PtracHistory} class which provides access to
    the \texttt{PtracEvent} class
\end{description}

Each of these three outputs will be discussed in more detail in the
following subsections.

\subsection{Accessing MCTAL Data with
MCNPTools}\label{accessing-mctal-data-with-mcnptools}

MCNP MCTAL file data is accessed via three of MCNPTools' classes:

\begin{description}
  \item[\texttt{Mctal} class] Provides object-oriented access to a MCTAL file.
  \item[\texttt{MctalTally} class] Provides object-oriented access to a tally in
    a MCTAL file
  \item[\texttt{MctalKcode} class] Provides object-oriented access to kcode
    outputs in a MCTAL file
\end{description}

Each class will be discussed in the following sections.

\subsubsection{\texttt{Mctal} Class}\label{the-mctal-class}

To construct (create) an instance of the \texttt{Mctal} class, one simply passes
the name of a MCTAL file to the \texttt{Mctal} constructor, e.g.,

\begin{lstlisting}[language=Python]
Mctal("mymctal")
\end{lstlisting}

The public methods available in the \texttt{Mctal} class are given in
Table~\ref{tab:mctal_class_public_methods}.

\begin{table}[]
  \begin{center}
  \caption{\texttt{Mctal} Class Public Methods}
  \label{tab:mctal_class_public_methods}
    \begin{tabular}{lp{4in}}
      \toprule
        Method & Description \\
      \midrule
        \texttt{GetCode()}      & Returns a string of the generating code name \\
        \texttt{GetVersion()}   & Returns a string of the code version \\
        \texttt{GetProbid()}    & Returns a string of the problem identification \\
        \texttt{GetDump()}      & Returns an integer of the corresponding restart dump number \\
        \texttt{GetNps()}       & Returns an integer of the number of histories used in the normalization \\
        \texttt{GetRandoms()}   & Returns an integer the number of random numbers used \\
        \texttt{GetTallyList()} & Returns a list/vector of tally numbers available in in the the MCTAL file \\
        \texttt{GetTally(NUM)}  & Returns a \texttt{MctalTally} class instance of tally number NUM \\
      \bottomrule
    \end{tabular}
  \end{center}
\end{table}

The most commonly used methods to access data in the MCTAL file are
\texttt{GetTallyList} and \texttt{GetTally} for tally data and \texttt{GetKcode}
for $k$-eigenvalue data. With \texttt{GetTallyList} and \texttt{GetTally}, loops
over the tallies in the MCTAL file can be created to perform analyses. A
Python example of such a loop structure is given in
Listing~\ref{lst:mctal_class_use_example}.

\begin{lstlisting}[
  language=Python,
  caption={\texttt{Mctal} Class Use Example},
  label={lst:mctal_class_use_example},
  float
]
# open the mctal file "mymctal"
mctal = mcnptools.Mctal("mymctal")

# loop over tallies
for tallynum in mctal.GetTallyList():
    tally = mctal.GetTally(tallynum)

    # now do something with the tally
\end{lstlisting}

\subsubsection{\texttt{MctalTally} Class}\label{the-mctaltally-class}

The \texttt{MctalTally} class should only be created through calls to the
\texttt{GetTally} method of the \texttt{Mctal} class. The \texttt{MctalTally}
class will provide information about the tally and the values of data contained
within the tally.

\textbf{A Note on MCNP Tallies}: MCNP tallies are essentially a nine-dimensional
array with each index of the array describing a bin structure of the tally.
These bin structures are given in Table~\ref{tab:mcnp_tally_array_indices}.

\begin{table}[]
  \begin{center}
  \caption{MCNP Tally Array Indices}
  \label{tab:mcnp_tally_array_indices}
    \begin{tabular}{llp{4in}}
      \toprule
        Name & Identifier & Description \\
      \midrule
        facet          &  \texttt{f}    & The facet of the tally, cell, surface, point number \\
        direct/flagged &  \texttt{d}    & The flagged/unflagged contribution for cell/surface tallies \emph{or} the direct/scattered contribution for point detectors (this dimension never exceeds 2) \\
        user           &  \texttt{u}    & The user bins established by use of an \texttt{FT} tally input or by use of a \texttt{TALLYX} routine \\
        segment        &  \texttt{s}    & The segmenting bins established by use of an \texttt{FS} tally input \\
        multiplier     &  \texttt{m}    & The multiplier bins established by use of an \texttt{FM} tally input \\
        cosine         &  \texttt{c}    & The cosine bins established by use of an \texttt{C} tally input \\
        energy         &  \texttt{e}    & The energy bins established by use of an \texttt{E} tally input \\
        time           &  \texttt{t}    & The time bins established by use of a \texttt{T} tally input \\
        perturbation   &  \texttt{pert} & The perturbation number established by use of \texttt{PERT} inputs \\
      \bottomrule
    \end{tabular}
  \end{center}
\end{table}

With these bin structures, the values and errors in a tally are uniquely
identified by the indices \texttt{(f,d,u,s,m,c,e,t,pert)}.

The \texttt{MctalTally} class has the public class methods given in
Table~\ref{tab:mctaltally_class_public_methods}.

\begin{table}[]
  \begin{center}
  \caption{\texttt{MctalTally} Class Public Methods}
  \label{tab:mctaltally_class_public_methods}
    \begin{tabular}{lp{3.5in}}
      \toprule
        Method & Description \\
      \midrule
        \texttt{ID()}                           & Return the integer tally number \\
        \texttt{GetFBins()}                     & Return a list/vector of the ``facet'' bins of the tally \\
        \texttt{GetDBins()}                     & Return a list/vector of the ``direct/flagged'' bins of the tally \\
        \texttt{GetUBins()}                     & Return a list/vector of the ``user'' bins of the tally \\
        \texttt{GetSBins()}                     & Return a list/vector of the ``segment'' bins of the tally \\
        \texttt{GetMBins()}                     & Return a list/vector of the ``multiplier'' bins of the tally \\
        \texttt{GetCBins()}                     & Return a list/vector of the ``cosine'' bins of the tally \\
        \texttt{GetEBins()}                     & Return a list/vector of the ``energy'' bins of the tally \\
        \texttt{GetTBins()}                     & Return a list/vector of the ``time'' bins of the tally \\
        \texttt{GetValue(f,d,u,s,m,c,e,t,pert)} & Return the tally value identified by the indices \texttt{(f,d,u,s,m,c,e,t,pert)} \\
        \texttt{GetValues(T,v1,v2,v3,v4,v5,v6,v7)} & Return the vector of values of specified bin \texttt{T} specifying other dimensions in order\\
        \texttt{GetError(f,d,u,s,m,c,e,t,pert)} & Return the tally \emph{relative} error identified by the indices \texttt{(f,d,u,s,m,c,e,t,pert)} \\
        \texttt{GetErrors(T,v1,v2,v3,v4,v5,v6,v7)} & Return the vector of errors of specified bin \texttt{T} specifying other dimensions in order\\
      \bottomrule
    \end{tabular}
  \end{center}
\end{table}

Often it is desirable to interrogate a tally value at the \emph{Tally
Fluctuation Chart} (TFC) bin---the bin on which statistical analyses are
performed. MCNPTools provides a defined constant \texttt{TFC} member of the
\texttt{MctalTally} class that can be used in place of a bin index for any of
the \texttt{(f,d,u,s,m,c,e,t)} bins. The Python code in
Listing~\ref{lst:mctaltally_class_use_example} illustrates how one would fill a
list with tally values by iterating over the energy bins of a tally (for brevity
it is assumed the MCTAL file has been opened as object \texttt{mctal}).

\begin{lstlisting}[
  language=Python,
  caption={\texttt{MctalTally} Class Use Example},
  label={lst:mctaltally_class_use_example},
  float
]
# get the tally of interest (say tally 834)
tally = mctal.GetTally(834)

# create an alias for the TFC bin
TFC = tally.TFC

# get the energy bins
ebins = tally.GetEBins()

#create lists for tally values and errors
values = list()
errors = list()

# iterate over the energy bins
for i in range( len(ebins) ):
    #                               f    d    u    s    m    c  e    t
    values.append( tally.GetValue(TFC, TFC, TFC, TFC, TFC, TFC, i, TFC) )
    errors.append( tally.GetError(TFC, TFC, TFC, TFC, TFC, TFC, i, TFC) )

# for large arrays the same operation can be performed faster with GetValues
values = tally.GetValues(tally.MctalTallyBins.E, TFC, TFC, TFC, TFC, TFC, TFC, TFC)
errors = tally.GetErrors(tally.MctalTallyBins.E, TFC, TFC, TFC, TFC, TFC, TFC, TFC)

\end{lstlisting}

Note that the \texttt{pert} index has been omitted from the example above. The
\texttt{GetValue} and \texttt{GetError} methods will default to the unperturbed
tally quantities if \texttt{pert} is omitted.

\subsubsection{\texttt{MctalKcode} Class}\label{the-mctalkcode-class}

The \texttt{MctalKcode} class should be obtained only through calls to
\texttt{GetKcode()} method of the \texttt{Mctal} class. The \texttt{MctalKcode}
class will provide information about the $k_{\mathrm{eff}}$ calculation as a
function of cycle. The \texttt{MctalKcode} class has the public methods given in
Table~\ref{tab:mctalkcode_class_public_methods}.

\begin{table}[]
  \begin{center}
  \caption{\texttt{MctalKcode} Class Public Methods}
  \label{tab:mctalkcode_class_public_methods}
    \begin{tabular}{lp{3.5in}}
      \toprule
        Method & Description \\
      \midrule
        \texttt{GetCycles()}               & return the integer number of total kcode cycles \\
        \texttt{GetSettle()}               & return the integer number of inactive kcode cycles \\
        \texttt{GetNdat()}                 & return the integer number of data elements in a kcode entry \\
        \texttt{GetValue(QUANTITY, CYCLE)} & return the value of \texttt{QUANTITY} at the specified \texttt{CYCLE} (default last) \\
      \bottomrule
    \end{tabular}
  \end{center}
\end{table}

The \texttt{QUANTITY} value that is passed into the \texttt{GetValue} method is
a parameterized member constant of the \texttt{MctalKcode} class.
\texttt{QUANTITY} must be one of the following defined parameters within the
\texttt{MctalKcode} class namespace as given in Table~\ref{tab:mctalkcode_quantity_values}.

\begin{table}[]
  \begin{center}
  \caption{\texttt{MctalKcode} Quantity Values}
  \label{tab:mctalkcode_quantity_values}
    \begin{tabular}{lp{4.0in}}
      \toprule
        Method & Description \\
      \midrule
        \texttt{COLLSION\_KEFF}                & estimated collision $k_{\mathrm{eff}}$ for this cycle \\
        \texttt{ABSORPTION\_KEFF}              & estimated absorption $k_{\mathrm{eff}}$ for this cycle \\
        \texttt{TRACKLENGTH\_KEFF}             & estimated track-length $k_{\mathrm{eff}}$ for this cycle \\
        \texttt{COLLISION\_PRLT}               & estimated collision prompt-removal lifetime for this cycle \\
        \texttt{ABSORPTION\_PRLT}              & estimated absorption prompt-removal lifetime for this cycle \\
        \texttt{AVG\_COLLSION\_KEFF}           & average collision $k_{\mathrm{eff}}$ to this cycle \\
        \texttt{AVG\_COLLSION\_KEFF\_STD}      & standard deviation in the collision $k_{\mathrm{eff}}$ to this cycle \\
        \texttt{AVG\_ABSORPTION\_KEFF}         & average absorption $k_{\mathrm{eff}}$ to this cycle \\
        \texttt{AVG\_ABSORPTION\_KEFF\_STD}    & standard deviation in the absorption $k_{\mathrm{eff}}$ to this cycle \\
        \texttt{AVG\_TRACKLENGTH\_KEFF}        & average track-length $k_{\mathrm{eff}}$ to this cycle \\
        \texttt{AVG\_TRACKLENGTH\_KEFF\_STD}   & standard deviation in the track-length $k_{\mathrm{eff}}$ to this cycle \\
        \texttt{AVG\_COMBINED\_KEFF}           & average combined $k_{\mathrm{eff}}$ to this cycle \\
        \texttt{AVG\_COMBINED\_KEFF\_STD}      & standard deviation in the combined $k_{\mathrm{eff}}$ to this cycle \\
        \texttt{AVG\_COMBINED\_KEFF\_BCS}      & average combined $k_{\mathrm{eff}}$ by cycles skipped \\
        \texttt{AVG\_COMBINED\_KEFF\_BCS\_STD} & standard deviation in the combined $k_{\mathrm{eff}}$ by cycles skipped \\
        \texttt{COMBINED\_PRLT}                & average combined prompt-removal lifetime \\
        \texttt{COMBINED\_PRLT\_STD}           & standard deviation in the combined prompt-removal lifetime \\
        \texttt{CYCLE\_NPS}                    & number of histories used in each cycle \\
        \texttt{AVG\_COMBINED\_FOM}            & combined figure of merit \\
      \bottomrule
    \end{tabular}
  \end{center}
\end{table}

The Python code in Listing~\ref{lst:mctalkcode_class_use_example} illustrates
how to get the combined (collision/absorption/track-length) value of
$k_{\mathrm{eff}}$ and its standard deviation (for brevity it is assumed the
MCTAL file has been opened in object \texttt{mctal}).

\begin{lstlisting}[
  language=Python,
  caption={\texttt{MctalKcode} Class Use Example},
  label={lst:mctalkcode_class_use_example},
  float
]
# get the kcode data from the mctal file
kcode = mctal.GetKcode()

# get the average combined keff from the last cycle
keff = kcode.GetValue(MctalKcode.AVG_COMBINED_KEFF)

# get the standard deviation in combined keff
keff = kcode.GetValue(MctalKcode.AVG_COMBINED_KEFF_STD)
\end{lstlisting}

\subsection{Accessing MESHTAL Data with MCNPTools}\label{accessing-meshtal-data-with-mcnptools}

MCNP column-formatted MESHTAL (type B, a.k.a, MCNP5 style mesh tallies from the
\texttt{fmesh} card) data is accessed through the following classes:

\begin{description}
  \item[\texttt{Meshtal}] provides object-oriented access to the MESHTAL file
  \item[\texttt{MeshtalTally}] provides object-oriented access to tally data
\end{description}

Each class will be discussed in the following sections.

\subsubsection{\texttt{Meshtal} Class}\label{the-meshtal-class}

To construct (create) an instance of the \texttt{Meshtal} class, one simply
passes the name of a MESHTAL (type B) file to the \texttt{Meshtal} constructor,
e.g.,

\begin{lstlisting}[language=Python]
Meshtal("mymeshtal")
\end{lstlisting}

The public methods available for the \texttt{Meshtal} class are given in
Table~\ref{tab:meshtal_class_public_methods}.

\begin{table}[]
  \begin{center}
  \caption{\texttt{Meshtal} Class Public Methods}
  \label{tab:meshtal_class_public_methods}
    \begin{tabular}{lp{3.5in}}
      \toprule
        Method & Description \\
      \midrule
        \texttt{GetCode()}      & return a string of the generating code name \\
        \texttt{GetVersion()}   & return a string the code version \\
        \texttt{GetProbid()}    & return a string the problem id number \\
        \texttt{GetComment()}   & return a string of the problem comment \\
        \texttt{GetNps()}       & return the number of histories to which values are normalized \\
        \texttt{GetTallyList()} & return a list/vector of tallies in the file \\
        \texttt{GetTally(NUM)}  & return a \texttt{MeshtalTally} class instance for tally \texttt{NUM} \\
      \bottomrule
    \end{tabular}
  \end{center}
\end{table}

The most commonly used methods of the \texttt{Meshtal} class are
\texttt{GetTallyList()} and \texttt{GetTally}. The Python code in
Listing~\ref{lst:meshtal_class_use_example} illustrates how to open a MESHTAL
file with the \texttt{Meshtal} class, loop over the tallies, and obtain the
tally data

\begin{lstlisting}[
  language=Python,
  caption={\texttt{Meshtal} Class Use Example},
  label={lst:meshtal_class_use_example},
  float
]
import mcnptools

# load the meshtal file mymeshtal
meshtal = mcnptools.Meshtal("mymeshtal")

# loop over all the tallies in the file
for tallynum in meshtal.GetTallyList():
    # obtain the tally data
    tally = meshtal.GetTally(tallynum)

    # now do something with the tally
\end{lstlisting}

\subsubsection{\texttt{MeshtalTally} Class}\label{the-meshtaltally-class}

The \texttt{MeshtalTally} provides accessors for a tally in a MESHTAL
file. The public methods of the \texttt{MeshtalTally} class are given in
Table~\ref{tab:meshtaltally_class_public_methods}.

\begin{table}[]
  \begin{center}
  \caption{\texttt{MeshtalTally} Class Public Methods}
  \label{tab:meshtaltally_class_public_methods}
    \begin{tabular}{lp{4.0in}}
      \toprule
        Method & Description \\
      \midrule
        \texttt{ID()}                & return a list/vector of the tally id (number) \\
        \texttt{GetXRBounds()}       & return a list/vector of the $x$/$r$ bin boundaries \\
        \texttt{GetYZBounds()}       & return a list/vector of the $y$/$z$ bin boundaries \\
        \texttt{GetZTBounds()}       & return a list/vector of the $z$/$\theta$ bin boundaries \\
        \texttt{GetEBounds()}        & return a list/vector of the energy bin boundaries \\
        \texttt{GetTBounds()}        & return a list/vector of the time bin boundaries \\
        \texttt{GetXRBins()}         & return a list/vector of the $x$/$r$ bin centers \\
        \texttt{GetYZBins()}         & return a list/vector of the $y$/$z$ bin centers \\
        \texttt{GetZTBins()}         & return a list/vector of the $z$/$\theta$ bin centers \\
        \texttt{GetEBins()}          & return a list/vector of the energy bins \\
        \texttt{GetTBins()}          & return a list/vector of the time bins \\
        \texttt{GetVolume(I,J,K)}    & return the volume of element at index \texttt{(I,J,K)} \\
        \texttt{GetValue(I,J,K,E,T)} & return the value at index \texttt{(I,J,K)} and optionally energy index \texttt{E} and time index \texttt{T} \\
        \texttt{GetError(I,J,K,E,T)} & return the \emph{relative} error at index \texttt{(I,J,K)} and optionally energy index \texttt{E} and time index \texttt{T} \\
      \bottomrule
    \end{tabular}
  \end{center}
\end{table}

If the energy-bin index is omitted from the \texttt{GetValue} or
\texttt{GetError} method calls, then the total bin will be used if present.
Otherwise, the largest energy bin will be used. Similarly, if the time-bin index
is omitted from the \texttt{GetValue} and \texttt{GetError} method calls then
the total bin will be used if present. Otherwise the last time bin will be used.

The Python code in Listing~\ref{lst:meshtaltally_class_use_example} illustrates
how to loop through spatial elements of a \texttt{MeshtalTally} and query the
values and errors at each element. For brevity it is assumed the MESHTAL file
has already been loaded into \texttt{meshtal}.

\begin{lstlisting}[
  language=Python,
  caption={\texttt{MeshtalTally} Class Use Example},
  label={lst:meshtaltally_class_use_example},
  float
]
# get the tally to process (e.g., tally 324)
tally = meshtal.GetTally(324)

xrbins = tally.GetXRBins()
yzbins = tally.GetYZBins()
ztbins = tally.GetZTBins()

# loop over xrbins
for i in range(len(xrbins)):
    # loop over yzbins
    for j in range(len(yzbins)):
        # loop over ztbins
        for k in range(len(ztbins)):
            # print the value and error
            print(i,j,k,meshtal.GetValue(i,j,k),meshtal.GetError(i,j,k))
\end{lstlisting}

\subsection{Accessing PTRAC Data with MCNPTools}\label{accessing-ptrac-data-with-mcnptools}

MCNP particle track (PTRAC) data are organized such that the PTRAC file contains
histories and each history contains events---i.e., things that actually happened
to particles. PTRAC data can be read and processed with MCNPTools by use of the
following classes:

\begin{description}
  \item[\texttt{Ptrac}] provides object-oriented access to PTRAC files and accesses \texttt{PtracHistory} classes
  \item[\texttt{PtracHistory}] provides object-oriented access to histories within the PTRAC file and accesses \texttt{PtracEvents}
  \item[\texttt{PtracNPS}] provides object-oriented access to NPS information in a \texttt{PtracHistory}
  \item[\texttt{PtracEvent}] provides object-oriented access to events and their data within a \texttt{PtracHistory}
\end{description}

The typical workflow when processing PTRAC files with MCNPTools is as follows:

\begin{enumerate}
  \item Open the PTRAC file with the \texttt{Ptrac} class
  \item Obtain histories in \texttt{PtracHistory} objects from the \texttt{Ptrac} class
  \item Iterate over the events in \texttt{PtracEvent} objects from the \texttt{PtracHistory} class
\end{enumerate}

Each of these classes is discussed in the sections that follow.

\subsubsection{\texttt{Ptrac} Class}\label{the-ptrac-class}

The \texttt{Ptrac} class opens and manages MCNP PTRAC files and supports legacy
binary, ASCII, and HDF5-formatted\footnote{HDF5-formatted PTRAC files are
anticipated to be available in the next public release of the MCNP code.} PTRAC
files. To construct the PTRAC file class, simply pass the PTRAC file name to the
\texttt{Ptrac} constructor with the file type. For example, in Python one would
use
\begin{lstlisting}
Ptrac("myptrac", Ptrac.BIN_PTRAC)
\end{lstlisting}
to open a legacy binary PTRAC file,
\begin{lstlisting}
Ptrac("myptrac", Ptrac.ASC_PTRAC)
\end{lstlisting}
to open an ASCII PTRAC file, and
\begin{lstlisting}
Ptrac("myptrac", Ptrac.HDF5_PTRAC)
\end{lstlisting}
to open an HDF5-formatted PTRAC file.

If the file type is omitted, legacy binary is assumed.

The \texttt{Ptrac} class has only one method \texttt{ReadHistories(NUM)} which
returns a list/vector of histories. If \texttt{NUM} is omitted, then all the
histories in the PTRAC file are read---this can be quite time consuming and is
generally not recommended. A typical to reading histories in Python is shown in
Listing~\ref{lst:ptrac_use_example}.

\begin{lstlisting}[
  language=Python,
  caption={\texttt{Ptrac} Class Use Example},
  label={lst:ptrac_use_example},
  float
]
# open the ptrac file (assuming legacy binary)
ptrac = mcnptools.Ptrac("myptrac")

# read history data in batches of 10000 histories
histories = ptrac.ReadHistories(10000)

# while histories has something in it
while histories:

    # iterate over the histories
    for h in histories:
        # do something with the history data

    # read in more histories, again a batch of 10000
    histories = ptrac.ReadHistories(10000)
\end{lstlisting}

\subsubsection{\texttt{PtracHistory} Class}\label{the-ptrachistory-class}

The \texttt{PtracHistory} class provides access to the events within the
history. The public class methods are given in
Table~\ref{tab:ptrachistory_class_public_methods}.

\begin{table}[]
  \begin{center}
  \caption{\texttt{PtracHistory} Class Public Methods}
  \label{tab:ptrachistory_class_public_methods}
    \begin{tabular}{lp{4.0in}}
      \toprule
        Method & Description \\
      \midrule
        \texttt{GetNPS()}       & returns a \texttt{PtracNPS} class with NPS information \\
        \texttt{GetNumEvents()} & returns the number of events in the history \\
        \texttt{GetEvent(I)}    & returns the \texttt{I}th event in the history \\
      \bottomrule
    \end{tabular}
  \end{center}
\end{table}

A typical use of the \texttt{PtracHistory} class to obtain its events using
Python is shown in Listing~\ref{lst:ptrachistory_use_example}, where it is
assumed that a \texttt{PtracHistory} exists in the variable \texttt{hist}):

\begin{lstlisting}[
  language=Python,
  caption={\texttt{PtracHistory} Class Use Example},
  label={lst:ptrachistory_use_example},
  float
]
for i in range(hist.GetNumEvents()):
    event = hist.GetEvent(i)

    # now do something with the event
\end{lstlisting}

\subsubsection{\texttt{PtracNPS} Class}\label{the-ptracnps-class}

The \texttt{PtracNPS} class contains information about the history. The
public methods in the \texttt{PtracNPS} class are given in
Table~\ref{tab:ptracnps_class_public_methods}.

\begin{table}[]
  \begin{center}
  \caption{\texttt{PtracNPS} Class Public Methods}
  \label{tab:ptracnps_class_public_methods}
    \begin{tabular}{lp{4.0in}}
      \toprule
        Method & Description \\
      \midrule
        \texttt{NPS()}     & return the history number \\
        \texttt{Cell()}    & return the filtering cell from CELL keyword (if present) \\
        \texttt{Surface()} & return the filtering surface from SURFACE keyword (if present) \\
        \texttt{Tally()}   & return the filtering tally from TALLY keyword (if present) \\
        \texttt{Value()}   & return the tally score from TALLY keyword (if present)  \\
      \bottomrule
    \end{tabular}
  \end{center}
\end{table}

For an HDF5 PTRAC file, the filtering cell, surface, tally, and value are not
recorded in the PTRAC file.  Please contact an MCNP developer at
\href{mailto:mcnp\_help@lanl.gov}{mcnp\_help@lanl.gov} if this limitation
proves prohibitive.

\subsubsection{\texttt{PtracEvent} Class}\label{the-ptracevent-class}

The \texttt{PtracEvent} class contains information about the event.
Different event types contain different information about the event. The
\texttt{PtracEvent} public class methods are given in
Table~\ref{tab:ptracevent_class_public_methods}.

\begin{table}[]
  \begin{center}
  \caption{\texttt{PtracEvent} Class Public Methods}
  \label{tab:ptracevent_class_public_methods}
    \begin{tabular}{lp{4.0in}}
      \toprule
        Method & Description \\
      \midrule
        \texttt{Type()}     & returns the event type: one of \texttt{Ptrac::SRC} (source), \texttt{Ptrac::BNK} (bank), \texttt{Ptrac::COL} (collision), \texttt{Ptrac::SUR} (surface crossing), or \texttt{Ptrac::TER} (termination) \\
        \texttt{BankType()} & returns the bank event type (only for \texttt{Ptrac::BNK} events) \\
        \texttt{Has(DATA)}  & returns a Boolean indicating whether or not the data type DATA is contained within the event \\
        \texttt{Get(DATA)}  & returns the value of the requested data type DATA  \\
      \bottomrule
    \end{tabular}
  \end{center}
\end{table}

The DATA types available for the \texttt{Has} and \texttt{Get} methods are part
of the \texttt{Ptrac} name space and are given in
Table~\ref{tab:ptracevent_data_types}.

\begin{table}[]
  \begin{center}
  \caption{\texttt{PtracEvent} Data Types}
  \label{tab:ptracevent_data_types}
    \begin{tabular}{lp{4.0in}}
      \toprule
        Data Type & Description \\
      \midrule
        \texttt{NODE}              & node number \\
        \texttt{ZAID}              & ZAID the particle interacts with \\
        \texttt{RXN}               & reaction type (MT number) \\
        \texttt{SURFACE}           & surface number \\
        \texttt{ANGLE}             & angle of particle crossing the surface \\
        \texttt{TERMINATION\_TYPE} & termination type \\
        \texttt{PARTICLE}          & particle type \\
        \texttt{CELL}              & cell number \\
        \texttt{MATERIAL}          & material number \\
        \texttt{COLLISION\_NUMBER} & collision number \\
        \texttt{X}                 & particle $x$ coordinate \\
        \texttt{Y}                 & particle $y$ coordinate \\
        \texttt{Z}                 & particle $z$ coordinate \\
        \texttt{U}                 & particle direction cosine with respect to the $x$ axis \\
        \texttt{V}                 & particle direction cosine with respect to the $y$ axis \\
        \texttt{W}                 & particle direction cosine with respect to the $z$ axis \\
        \texttt{ENERGY}            & particle energy \\
        \texttt{WEIGHT}            & particle weight \\
        \texttt{TIME}              & particle time  \\
      \bottomrule
    \end{tabular}
  \end{center}
\end{table}

The Python code given in Listing~\ref{lst:ptracevent_use_example} demonstrates
how to find all collision events in a history and print the energy (for brevity
a \texttt{PtracHistory} instance is assumed to be in the \texttt{hist}
variable).

\begin{lstlisting}[
  language=Python,
  caption={\texttt{PtracEvent} Class Use Example},
  label={lst:ptracevent_use_example},
  float
]
#iterate over all events in the history
for i in range(hist.GetNumEvents()):
    event = hist.GetEvent()

    # check if the event is a collision event
    if( event.Type() == Ptrac.COL ):
        # print the energy
        print(event.Get(Ptrac.ENERGY))
\end{lstlisting}

The PTRAC bank type variable specifiers that are part of the \texttt{Ptrac} name
space are listed in Table~\ref{tab:ptracevent_data_types}.

\begin{table}[]
  \begin{center}
  \caption{\texttt{PtracEvent} Data Types}
  \label{tab:ptracevent_data_types}
    \begin{tabular}{lp{4.75in}}
      \toprule
        Data Type & Description \\
      \midrule
        \texttt{BNK\_DXT\_TRACK}       & DXTRAN particle \\
        \texttt{BNK\_ERG\_TME\_SPLIT}  & Energy or Time splitting \\
        \texttt{BNK\_WWS\_SPLIT}       & Weight-window surface crossing \\
        \texttt{BNK\_WWC\_SPLIT}       & Weight-window collision \\
        \texttt{BNK\_UNC\_TRACK}       & Forced-collision uncollided part \\
        \texttt{BNK\_IMP\_SPLIT}       & Importance splitting \\
        \texttt{BNK\_N\_XN\_F}         & Neutrons from fission \\
        \texttt{BNK\_N\_XG}            & Gammas from neutron production \\
        \texttt{BNK\_FLUORESCENCE}     & Fluorescence x-rays \\
        \texttt{BNK\_ANNIHILATION}     & Annihilation photons \\
        \texttt{BNK\_PHOTO\_ELECTRON}  & Photo electrons \\
        \texttt{BNK\_COMPT\_ELECTRON}  & Compton electrons \\
        \texttt{BNK\_PAIR\_ELECTRON}   & Pair-production electron \\
        \texttt{BNK\_AUGER\_ELECTRON}  & Auger electrons \\
        \texttt{BNK\_PAIR\_POSITRON}   & Pair-production positron \\
        \texttt{BNK\_BREMSSTRAHLUNG}   & Bremsstrahlung production \\
        \texttt{BNK\_KNOCK\_ON}        & Knock-on electron \\
        \texttt{BNK\_K\_X\_RAY}        & K-shell x-ray production \\
        \texttt{BNK\_N\_XG\_MG}        & Multigroup (n,x$\gamma$) \\
        \texttt{BNK\_N\_XF\_MG}        & Multigroup (n,f) \\
        \texttt{BNK\_N\_XN\_MG}        & Multigroup (n,xn) \\
        \texttt{BNK\_G\_XG\_MG}        & Multigroup ($\gamma$,x$\gamma$) \\
        \texttt{BNK\_ADJ\_SPLIT}       & Multigroup adjoint splitting \\
        \texttt{BNK\_WWT\_SPLIT}       & Weight-window mean-free-path split \\
        \texttt{BNK\_PHOTONUCLEAR}     & Photo-nuclear production \\
        \texttt{BNK\_DECAY}            & Radioactive decay \\
        \texttt{BNK\_NUCLEAR\_INT}     & Nuclear interaction \\
        \texttt{BNK\_RECOIL}           & Recoil nucleus \\
        \texttt{BNK\_DXTRAN\_ANNIHIL}  & DXTRAN annihilation photon from pulse-height tally variance reduction \\
        \texttt{BNK\_N\_CHARGED\_PART} & Light ions from neutrons \\
        \texttt{BNK\_H\_CHARGED\_PART} & Light ions from protons \\
        \texttt{BNK\_N\_TO\_TABULAR}   & Library neutrons from model neutrons \\
        \texttt{BNK\_MODEL\_UPDAT1}    & Secondary particles from inelastic nuclear interactions \\
        \texttt{BNK\_MODEL\_UPDATE}    & Secondary particles from elastic nuclear interactions \\
        \texttt{BNK\_DELAYED\_NEUTRON} & Delayed neutron from radioactive decay \\
        \texttt{BNK\_DELAYED\_PHOTON}  & Delayed photon from radioactive decay \\
        \texttt{BNK\_DELAYED\_BETA}    & Delayed $\beta^-$ from radioactive decay \\
        \texttt{BNK\_DELAYED\_ALPHA}   & Delayed $\alpha$ from radioactive decay \\
        \texttt{BNK\_DELAYED\_POSITRN} & Delayed $\beta^+$ from radioactive decay \\
      \bottomrule
    \end{tabular}
  \end{center}
\end{table}

The PTRAC termination types that are members of the \texttt{Ptrac} name space
are listed in Table~\ref{tab:ptracevent_termination_types}.

\begin{table}[]
  \begin{center}
  \caption{\texttt{PtracEvent} Termination Types}
  \label{tab:ptracevent_termination_types}
    \begin{tabular}{lp{3.5in}}
      \toprule
        Termination Type & Description \\
      \midrule
        \texttt{TER\_ESCAPE}                        & Escape \\
        \texttt{TER\_ENERGY\_CUTOFF}                & Energy cutoff \\
        \texttt{TER\_TIME\_CUTOFF}                  & Time cutoff \\
        \texttt{TER\_WEIGHT\_WINDOW}                & Weight-window roulette \\
        \texttt{TER\_CELL\_IMPORTANCE}              & Cell importance roulette \\
        \texttt{TER\_WEIGHT\_CUTOFF}                & Weight-cutoff roulette \\
        \texttt{TER\_ENERGY\_IMPORTANCE}            & Energy-importance roulette \\
        \texttt{TER\_DXTRAN}                        & DXTRAN roulette \\
        \texttt{TER\_FORCED\_COLLISION}             & Forced-collision \\
        \texttt{TER\_EXPONENTIAL\_TRANSFORM}        & Exponential-transform \\
        \texttt{TER\_N\_DOWNSCATTERING}             & Neutron downscattering \\
        \texttt{TER\_N\_CAPTURE}                    & Neutron capture \\
        \texttt{TER\_N\_N\_XN}                      & Loss to (n,xn) \\
        \texttt{TER\_N\_FISSION}                    & Loss to fission \\
        \texttt{TER\_N\_NUCLEAR\_INTERACTION}       & Nuclear interactions \\
        \texttt{TER\_N\_PARTICLE\_DECAY}            & Particle decay \\
        \texttt{TER\_N\_TABULAR\_BOUNDARY}          & Tabular boundary \\
        \texttt{TER\_P\_COMPTON\_SCATTER}           & Photon Compton scattering \\
        \texttt{TER\_P\_CAPTURE}                    & Photon capture \\
        \texttt{TER\_P\_PAIR\_PRODUCTION}           & Photon pair production \\
        \texttt{TER\_P\_PHOTONUCLEAR}               & Photonuclear reaction \\
        \texttt{TER\_E\_SCATTER}                    & Electron scatter \\
        \texttt{TER\_E\_BREMSSTRAHLUNG}             & Bremsstrahlung \\
        \texttt{TER\_E\_INTERACTION\_DECAY}         & Interaction or decay \\
        \texttt{TER\_GENNEUT\_NUCLEAR\_INTERACTION} & Generic neutral-particle nuclear interactions \\
        \texttt{TER\_GENNEUT\_ELASTIC\_SCATTER}     & Generic neutral-particle elastic scatter \\
        \texttt{TER\_GENNEUT\_DECAY}                & Generic neutral-particle particle decay \\
        \texttt{TER\_GENCHAR\_MULTIPLE\_SCATTER}    & Generic charged-particle multiple scatter \\
        \texttt{TER\_GENCHAR\_BREMSSTRAHLUNG}       & Generic charged-particle bremsstrahlung \\
        \texttt{TER\_GENCHAR\_NUCLEAR\_INTERACTION} & Generic charged-particle nuclear interactions \\
        \texttt{TER\_GENCHAR\_ELASTIC\_SCATTER}     & Generic charged-particle elastic scatter \\
        \texttt{TER\_GENCHAR\_DECAY}                & Generic charged-particle particle decay \\
        \texttt{TER\_GENCHAR\_CAPTURE}              & Generic charged-particle capture \\
        \texttt{TER\_GENCHAR\_TABULAR\_SAMPLING}    & Generic charged-particle tabular sampling \\
      \bottomrule
    \end{tabular}
  \end{center}
\end{table}

\clearpage
\section{Acknowledgments}\label{acknowledgments}

The authors acknowledge Mike Rising, David Dixon, and Jeff Bull for their review
of MCNPTools documentation and for testing it.
The authors also appreciate user feedback such as the patch proposed by Matthew
T.~Montgomery to fix a \texttt{mctal}-file parsing error.
Finally, the authors are grateful to the support provided
by the Advanced Simulation and Computing (ASC) Program to develop, maintain, and
release MCNPTools.

\newpage \appendix

\section{Help Messages for MCNPTools Utilities}\label{Help Messages for MCNPTools Utilities}

\lstinputlisting[
  caption={\texttt{l3d2vtk} Help Message Output},
  label={lst:l3d2vtk-help-message-output},
]{include/help_messages/help_l3d2vtk.txt}

\clearpage
\lstinputlisting[
  caption={\texttt{l3dcoarsen} Help Message Output},
  label={lst:l3dcoarsen-help-message-output},
]{include/help_messages/help_l3dcoarsen.txt}

\clearpage
\lstinputlisting[
  caption={\texttt{l3dinfo} Help Message Output},
  label={lst:l3dinfo-help-message-output},
]{include/help_messages/help_l3dinfo.txt}

\clearpage
\lstinputlisting[
  caption={\texttt{l3dscale} Help Message Output},
  label={lst:l3dscale-help-message-output},
]{include/help_messages/help_l3dscale.txt}

\clearpage
\lstinputlisting[
  caption={\texttt{mctal2rad} Help Message Output},
  label={lst:mctal2rad-help-message-output},
]{include/help_messages/help_mctal2rad.txt}

\clearpage
\lstinputlisting[
  caption={\texttt{mergemctals} Help Message Output},
  label={lst:mergemctals-help-message-output},
]{include/help_messages/help_mergemctals.txt}

\clearpage
\lstinputlisting[
  caption={\texttt{mergemeshtals} Help Message Output},
  label={lst:mergemeshtals-help-message-output},
]{include/help_messages/help_mergemeshtals.txt}

\clearpage
\lstinputlisting[
  caption={\texttt{meshtal2vtk} Help Message Output},
  label={lst:meshtal2vtk-help-message-output},
]{include/help_messages/help_meshtal2vtk.txt}

\clearpage
\section{C++ Examples}\label{c-examples}

\subsection{Mctal Example 1}\label{mctal-example-1}

Listing~\ref{lst:C++ Mctal Example 1} opens the MCTAL file
\texttt{example\_mctal\_1.mcnp.mctal} and extracts the energy bins and
energy-bin tally values for tally 4.

\lstinputlisting[
  caption={C++ Mctal Example 1},
  label={lst:C++ Mctal Example 1},
  language=C++
]{include/installation_and_use_examples/cpp_example_mctal_1.cpp}
\embedfile[ucfilespec={cpp\_example\_mctal\_1.cpp.txt}]{include/installation_and_use_examples/cpp_example_mctal_1.cpp}

\clearpage
\subsection{Mctal Example 1 with GetValues}\label{mctal-example-1-getvalues}

Listing~\ref{lst:C++ Mctal Example 1 GetValues} opens the MCTAL file
\texttt{example\_mctal\_1.mcnp.mctal} and extracts the
energy-bin tally values for tally 4 using the GetValues method.

\lstinputlisting[
  caption={C++ Mctal Example 1 GetValues},
  label={lst:C++ Mctal Example 1 GetValues},
  language=C++
]{include/installation_and_use_examples/cpp_example_mctal_1_getvalues.cpp}
\embedfile[ucfilespec={cpp\_example\_mctal\_1\_getvalues.cpp.txt}]{include/installation_and_use_examples/cpp_example_mctal_1_getvalues.cpp}

\clearpage
\subsection{Mctal Example 2}\label{mctal-example-2}

Listing~\ref{lst:C++ Mctal Example 2} opens the MCTAL file
\texttt{example\_mctal\_2.mcnp.mctal} and extracts the $k_{\mathrm{eff}}$ value
and standard deviation for the active cycles, i.e., from the last settle cycle
through the last active cycle.

\lstinputlisting[
  caption={C++ Mctal Example 2},
  label={lst:C++ Mctal Example 2},
  language=C++
]{include/installation_and_use_examples/cpp_example_mctal_2.cpp}
\embedfile[ucfilespec={cpp\_example\_mctal\_2.cpp.txt}]{include/installation_and_use_examples/cpp_example_mctal_2.cpp}

\clearpage
\subsection{Meshtal Example}\label{meshtal-example}

Listing~\ref{lst:C++ Meshtal Example} reads tally 4 from MESHTAL file
\texttt{example\_meshtal.mcnp.meshtal} and prints the values at a slice through
the $z$ index 5 (using 0 indexing).

\lstinputlisting[
  caption={C++ Meshtal Example},
  label={lst:C++ Meshtal Example},
  language=C++
]{include/installation_and_use_examples/cpp_example_meshtal.cpp}
\embedfile[ucfilespec={cpp\_example\_meshtal.cpp.txt}]{include/installation_and_use_examples/cpp_example_meshtal.cpp}

\clearpage
\subsection{Ptrac Example 1}\label{ptrac-example-1}

Listing~\ref{lst:C++ Ptrac Example 1} opens the binary PTRAC file
\texttt{example\_ptrac\_1.mcnp.ptrac} and prints the $\left(x,y,z\right)$
location and energy of bank events.

\lstinputlisting[
  caption={C++ Ptrac Example 1},
  label={lst:C++ Ptrac Example 1},
  language=C++
]{include/installation_and_use_examples/cpp_example_ptrac_1.cpp}
\embedfile[ucfilespec={cpp\_example\_ptrac\_1.cpp.txt}]{include/installation_and_use_examples/cpp_example_ptrac_1.cpp}

\clearpage
\subsection{Ptrac Example 2}\label{ptrac-example-2}

Listing~\ref{lst:C++ Ptrac Example 2} opens binary PTRAC file
\texttt{example\_ptrac\_2.mcnp.ptrac} and prints the $\left(x,y,z\right)$
location and angle of surface crossings.

\lstinputlisting[
  caption={C++ Ptrac Example 2},
  label={lst:C++ Ptrac Example 2},
  language=C++
]{include/installation_and_use_examples/cpp_example_ptrac_2.cpp}
\embedfile[ucfilespec={cpp\_example\_ptrac\_2.cpp.txt}]{include/installation_and_use_examples/cpp_example_ptrac_2.cpp}

\clearpage
\section{Python Examples}\label{python-examples}

\subsection{Mctal Example 1}\label{mctal-example-1-1}

Listing~\ref{lst:Python Mctal Example 1} opens the MCTAL file
\texttt{example\_mctal\_1.mcnp.mctal} and extracts the energy bins and
energy-bin tally values for tally 4.

\lstinputlisting[
  caption={Python Mctal Example 1},
  label={lst:Python Mctal Example 1},
  language=Python
]{include/installation_and_use_examples/python_example_mctal_1.py}
\embedfile[ucfilespec={python\_example\_mctal\_1.py.txt}]{include/installation_and_use_examples/python_example_mctal_1.py}

\clearpage
\subsection{Mctal Example 2}\label{mctal-example-2-1}

Listing~\ref{lst:Python Mctal Example 2} opens the MCTAL file
\texttt{example\_mctal\_2.mcnp.mctal} and extracts the $k_{\mathrm{eff}}$ value
and standard deviation for the active cycles, i.e., from the last settle cycle
through the last active cycle.

\lstinputlisting[
  caption={Python Mctal Example 2},
  label={lst:Python Mctal Example 2},
  language=Python
]{include/installation_and_use_examples/python_example_mctal_2.py}
\embedfile[ucfilespec={python\_example\_mctal\_2.py.txt}]{include/installation_and_use_examples/python_example_mctal_2.py}

\clearpage
\subsection{Meshtal Example}\label{meshtal-example-1}

Listing~\ref{lst:Python Meshtal Example} reads tally 4 from MESHTAL file
\texttt{example\_meshtal.mcnp.meshtal} and prints the values at a slice through
the $z$ index 5 (using 0 indexing).

\lstinputlisting[
  caption={Python Meshtal Example},
  label={lst:Python Meshtal Example},
  language=Python
]{include/installation_and_use_examples/python_example_meshtal.py}
\embedfile[ucfilespec={python\_example\_meshtal.py.txt}]{include/installation_and_use_examples/python_example_meshtal.py}

\clearpage
\subsection{Ptrac Example 1}\label{ptrac-example-1-1}

Listing~\ref{lst:Python Ptrac Example 1} opens the legacy binary PTRAC file
\texttt{example\_ptrac\_1.mcnp.ptrac} and prints the $\left(x,y,z\right)$
location and energy of bank events.

\lstinputlisting[
  caption={Python Ptrac Example 1},
  label={lst:Python Ptrac Example 1},
  language=Python
]{include/installation_and_use_examples/python_example_ptrac_1.py}
\embedfile[ucfilespec={python\_example\_ptrac\_1.py.txt}]{include/installation_and_use_examples/python_example_ptrac_1.py}

\clearpage
\subsection{Ptrac Example 2}\label{ptrac-example-2-1}

Listing~\ref{lst:Python Ptrac Example 2} opens legacy binary PTRAC file
\texttt{example\_ptrac\_2.mcnp.ptrac} and prints the $\left(x,y,z\right)$
location and angle of surface crossings.

\lstinputlisting[
  caption={Python Ptrac Example 2},
  label={lst:Python Ptrac Example 2},
  language=Python
]{include/installation_and_use_examples/python_example_ptrac_2.py}
\embedfile[ucfilespec={python\_example\_ptrac\_2.py.txt}]{include/installation_and_use_examples/python_example_ptrac_2.py}

\clearpage
\subsection{Ptrac Example 3}\label{ptrac-example-3-1}

Listing~\ref{lst:Python Ptrac Example 3} opens HDF5 PTRAC file
\texttt{example\_ptrac\_3.mcnp.ptrac.h5} and prints information about
surface-crossing and termination events.

\lstinputlisting[
  caption={Python Ptrac Example 3},
  label={lst:Python Ptrac Example 3},
  language=Python
]{include/installation_and_use_examples/python_example_ptrac_3.py}
\embedfile[ucfilespec={python\_example\_ptrac\_3.py.txt}]{include/installation_and_use_examples/python_example_ptrac_3.py}

\end{document}

